\documentclass[letterpaper,10pt,twoside,onecolumn,openany]{book}
\usepackage[bg=full,layout=true]{dnd}
\usepackage{dndnotes}

\begin{document}
\tableofcontents

\chapter{This is a chapter}

\section{This is a section}
\lipsum[1]
\begin{quotebox}
    This is a quote box useful for quotes the default color is this.
\end{quotebox}
\lipsum[11]\Cues{This is a Cue, useful in the context for cornell notes}
\subsection{This is a subsection}
\begin{Theorem}{Pythagous Theorem}
\label{pythagorean}
This is a theorem about right triangles and can be summarized in the next equation 
\[ x^2 + y^2 = z^2 \]
\Unsure{Not too sure if this is correct, ask the prof}
And a consequence of theorem \ref{pythagorean} is the statement in the next corollary.
\end{Theorem}
\lipsum[5]
\begin{Note}
    \lipsum[4]
    \lipsum[6]
\end{Note}
\subsubsection{This is subsubsection}
\begin{Questions}
    \item This is the first question 
    \item This is the second question
\end{Questions} 
\begin{Answers}
    \item This is the first answer
    \item This is the second answer
\end{Answers}
\( \varepsilon \in \mathbb{R} \) 
\begin{Definition}{The average}
    The average of a function between a range is such
    \[f_{avg} = \frac{1}{b-a} \int_{a}^{b}f(x)dx\]
    \Info{This can be helpful at the end}
    We can rearrange this to get this:
    \[\int_{a}^{b}f(x)dx = f(c)(b-a)\]
\end{Definition}
\lipsum[10]
\begin{Corollary}
    There's no right rectangle whose sides measure 3cm, 4cm, and 6cm.
\end{Corollary}
\begin{Lemma}
    Given two line segments whose lengths are $a$ and $b$ respectively there is a 
real number $r$ such that $b=ra$.
\end{Lemma}
\begin{Review}
    This is a review box useful for reviewing at the end of a lecture
    \begin{itemize}
        \item This is a Review on what is going on
        \item So far this is what it would look like
    \end{itemize}
\end{Review}
\emph{This is really important to understand as it is going to be tested on the exam}
\section{This is another section}
\begin{Definition}{Test Def}
    This is a test definition
\end{Definition}
\begin{paperbox}{}
    \begin{theorem}[Pythagous theorem ]
        Is a theorem where a right angle triangle has this following properties
        \[a^2 + b^2 = c^2\]
        Therefore the right angle triangle will have a max degree of something or other.
    \end{theorem}
\end{paperbox}
\Improve{There is a better way at writing this}
Hello world this is jihoon speak from the internet
what the fuck is wrong with this symbol of a computer and that is not working well and is not really
\(\int_{}^{}\Sigma_{f \to \epsilon   }   \frac{\sigma}{\nu}\rho\)
\lipsum[2]
\underline{}
\lipsum[3]
\begin{quotebox}
    \lipsum[1]
\end{quotebox}
\lipsum[4]
\lipsum[5]
\lipsum[6]
\lipsum[7]
\newpage
limit of things that are not responsible for the action of the paperwork are
\begin{proof}
    Let us assume that the entire graph structure is even. Therefore, we can assume that this is possible.
    \[R \in \mathcal{R} \rightarrow \subset \mathcal{Q}\]
\end{proof}
\begin{equation}
    L = \int_{a}^{b}\sqrt{1 + (f'(x))^2}dx
\end{equation}
\begin{equation}
    SA  =\int_{a}^{b}2\pi f(x) ds
\end{equation}
\begin{equation}
    ds = \sqrt{1 + (f'(x))}dx   
\end{equation}
\begin{equation}
    SA = 2\pi \int_{a}^{b} f(x) \sqrt{1 + (f'(x))}dx
\end{equation}
\begin{paperbox}{Title}[PhbLightCyan]
    Body
\end{paperbox}
\begin{paperbox}{Title}[PhbMauve]
    Body
\end{paperbox}
\begin{paperbox}{Title}[PhbTan]
    Body
\end{paperbox}
\begin{paperbox}{Title}[DmgLavender]
    Body
\end{paperbox}
\begin{paperbox}{Title}[DmgCoral]
    Body
\end{paperbox}
\begin{paperbox}{Title}[DmgSlateGray]
    Body
\end{paperbox}
\begin{paperbox}{Title}[DmgLilac]
    Body
\end{paperbox}



\end{document}