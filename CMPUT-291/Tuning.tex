\documentclass{article}

\begin{document}
\section{Measure of Performance}
\begin{description}
    \item [Response time]: Average time to wait for a response to a particular query
    \item [Throughput]: volume of work completed in a fixed amount of time (measured in transactions per second)
    \item [Workload]: The amount and priority of queries processed by the database. (Lower the better)
\end{description}
\section{Normalization}
Can improve or decrease performance.\\
It can improve performance as less redundancy leads to more unique rows per record leading to a decrease in I/O operation.\\
More tables leads to smaller and more clustered indexes.\\
However, it can decrease performance as reducing redundancies can leads to a loss in functional dependencies which often leads to costly join operations.
There is a trade off between maintaining functional dependencies and reducing redundancies. The choice of optimization is dependent on the situation at hand.
\end{document}